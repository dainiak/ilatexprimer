\section{\(\LaTeX\) — это язык}
\index{LaTeX}
\par \LaTeX --- это, по сути, язык разметки. Вроде HTML, только куда более мощный (поскольку поддерживает макро-программирование). Вы пишете на этом языке исходный файл, а специальный компилятор \LaTeX, подумав, создаёт по Вашему исходному тексту красивый PDF-документ (раньше формат вывода был не PDF, а DVI или PS, но последние фактически устаревают).

\par Язык \LaTeX и его более ранняя версия \TeX, созданная живым классиком информатики \href{https://en.wikipedia.org/wiki/Donald_Knuth}{Дональдом Кнутом}, стали стандартом среди математического и естественно-научного сообщества, поскольку не имеют конкурентов в гибкости, переносимости и распространённости. Например, все современные версии Microsoft Word и Microsoft PowerPoint (начиная с 2007) поддерживают ввод формул в формате \LaTeX. В интернет-сообществе широко известна мощная система \href{https://en.wikipedia.org/wiki/MathJax}{MathJax}\index{MathJax}, которая <<на лету>> отображает формулы, записанные в формате \LaTeX на веб-страницах. Кстати, именно эта система используется в настоящем интерактивном уроке.

\par Конечно, поскольку \LaTeX не является \href{https://en.wikipedia.org/wiki/WYSIWYG}{WYSIWYG-системой}, есть одна трудность: чтобы его использовать, нужно пройти некоторое обучение. Однако, по опыту, для того, чтобы набирать на \LaTeX б\'{о}льшую часть повседневных формул и целых текстов, достаточно потратить не более двух-трёх часов на первоначальное знакомство, а затем просто иметь перед глазами \href{https://wch.github.io/latexsheet/}{шпаргалку} с описанием популярных команд системы. По \LaTeX написано множество руководств, поэтому при желании вполне возможно изучить \LaTeX во всей полноте, и тогда можно дополнять систему своими собственными макрокомандами и библиотеками (модулями).


\section{Дословное воспроизведение}
\par Иногда нужно вывести текст дословно, так, чтобы \LaTeX его никак не обрабатывал. В этом случае применяют команду \verb"\verb"\index{\verb}, после которой в кавычках можно написать выводимую последовательность. Это крайне полезно, учитывая, что \LaTeX делает довольно много замен: например, повторенные символы \verb"<<…>>"\index{кавычки} преобразуются в <<кавычки>>, а \verb"---"\index{тире} --- в длинное тире. А как вывести символ кавычки внутри \verb"\verb"? Ответ: нужно написать, например, \verb+\verb|кавычка: "|+. (А теперь посмотрите на код справа.)\index{\verb=дословное цитирование=verbatim}
\par Особенно осторожными нужно быть в \LaTeX со следующими символами: \verb"& { } $ % # _". Скажем, символ \%, если перед ним нет обратного слеша, является началом комментария, который будет пропущен системой при компиляции.\index{комментарии}%Как, например, этот комментарий в коде, который Вы вряд ли увидите слева.


\section{Параграфы}
\par Текст обычно разбивают на параграфы. Каждый новый параграф начинается командой \verb"\par"\index{\par} или если \LaTeX встречает внутри текста (не формулы!) два перехода на новую строку подряд. Последний способ в данном уроке не поддерживается.\index{\par=начать новый параграф}
\par Помимо создания нового параграфа, вывести текст с новой строки можно и другими способами, но о них --- позже.


\section{Выделение}
\par Чтобы выделять текст, используются команды \verb"\textbf"\index{\textbf=полужирный шрифт=жирный шрифт} (\textbf{жирный} шрифт) и \verb"\textit"\index{\textit=курсив} (\textit{курсивный} шрифт), которые можно \textbf{комби\textit{нир}овать}. Также есть команда для подчёркивания \verb"\underline", но в большинстве современных руководств по оформлению печатных текстов рекомендуют не использовать подчёркивание для выделения текста, а потому эта команда не поддерживается настоящим интерактивным уроком;)

\par Есть ещё одна замечательная команда \verb"\emph"\index{\emph=выделение текста}, которая старается сделать текст выделяющимся, независимо от того, какие шрифты используются вокруг: скажем, во фрагменте кода \verb"\textit{это было прошлым \emph{летом}, в середине \emph{января}}" слова <<января>> и <<летом>> выведутся обычным шрифтом, чтобы отличаться от окружающих их слов, набранных курсивом. Поэтому при отсутствии каких-либо специальных мотивов текст рекомендуется выделять именно с помощью \verb"\emph". Внимание: в данном уроке команда \verb"\emph" не так продвинута, и в полной красе её действие видно только в полноценном \LaTeX.

\par Иногда нужно выделить не слово, а отдельную букву: чаще всего речь идёт об ударении. К счастью, в \LaTeX без труда можно <<навесить>> ударение над буквой с помощью сочетания \verb"\'{о}"\index{диакритические знаки,ударение} (вместо о может быть люб\'{а}я б\'{у}ква). Это частный случай использования диакритических знаков, полный список команд для них можно посмотреть по \href{https://en.wikibooks.org/wiki/LaTeX/Special_Characters#Escaped_codes}{ссылке}.

\par Пока что это всё, что для начала нужно знать об оформлении текста. Главное всё-таки --- это формулы, о них и пойдёт речь в непосредственно следующих шагах.


\section{Два основных режима: текстовый и математический}
\par Хотя в \LaTeX вполне можно верстать книги, не содержащие ни одной формулы, чаще всего эту систему используют именно благодаря непревзойдённому качеству отображения математической нотации. Первое, что нужно знать о наборе в \LaTeX --- это наличие двух режимов: \textbf{обычного (текстового)} и \textbf{математического (формульного)}. Как правило, когда Вы пишете математический текст, Вы можете однозначно сказать про каждый символ, имеет ли он стандартное языковое значение или это математическое обозначение. Чтобы \LaTeX распознал, что именно Вы считаете формулой, нужно заключать формулу в доллары, например: $F=ma$ --- это формула, заключённая в доллары, и \LaTeX вывел её особым шрифтом, а F=ma --- формула, которую в доллары заключить забыли, и она выведена как обычный текст, тем же текстовым шрифтом. Вместо долларов \verb"$…$" можно использовать сочетание \verb"\(…\)", и результат будет точно таким же: \(F=ma.\) Рекомендуется пользоваться именно последним способом оформления, поскольку здесь явно видно начало формулы и её конец, в то время как <<открывающий>> и <<закрывающий>> доллар \LaTeX-у приходится угадывать самостоятельно. Однако, эта рекомендация не является строгой.


\section{Два основных математических режима: inline и display}
\par Как и в хорошо набранных книгах, в \LaTeX можно как располагать формулы внутри текста, так и делать их центром внимания, вынося на отдельную строчку (такие формулы называются \textit{выключными} или \textit{выносными}). Когда формула располагается в окружении текста, такой математический режим называется inline math, а когда формула располагается на отдельной строчке --- display math. Чтобы формула была вынесена на середину отдельной строки, её заключают в двойные доллары: \verb"$$…$$" в системе $\TeX$ и в квадратные <<бэкслэшнутые>> скобки \verb"\[…\]", --- в \LaTeX. Вот пример: \[F=ma.\]
В отличие от чисто виртуального различия между записью \verb"$…$" и \verb"\(…\)" для inline math, разница записей \verb"$$…$$" и \verb"\[…\]" существенна: хотя часто разницы не видно, вполне может возникнуть ситуация, когда формула, оформленная двойными долларами, имеет отступы от окружающего текста не такие, как задумывалось.\index{дисплейный режим=display math}


\section{Упражнение}
\begin{staticpart}
Наберите следующий текст, сохраняя разбиение на абзацы, шрифт текста и положение формул (для удобства неоформленный текст уже дан Вам в окне кода):
\htmlblockquote{\par Совсем нетрудно набирать тексты, содержащие \textbf{жирный шрифт}, а иногда и \textit{курсив}, хотя выделять текст лучше всё же командой \verb"\emph".
\par Разбивать текст на параграфы, ставить <<кавычки-ёлочки>> --- дело ТеХники! Но вот что лучше всего, так это набор формул. Ведь так приятно написать, что \(\showSourceOnClick A+B=B+A\) или даже заявить всему миру, что \[\showSourceOnClick C+D=D+C.\]}
\end{staticpart}
Совсем нетрудно набирать тексты, содержащие жирный шрифт, а иногда и курсив, хотя выделять текст лучше всё же командой emph. Разбивать текст на параграфы, ставить кавычки-ёлочки дело ТеХники! Но вот что лучше всего, так это набор формул. Ведь так приятно написать, что A+B=B+A или даже заявить всему миру, что C+D=D+C.


\section{Списки}
\par Помимо команд в \LaTeX есть ещё понятие \emph{окружений}.\index{окружение} Окружения задаются с помощью пары команд вида \verb"\begin{…}" и \verb"\end{…}". Позже мы познакомимся со многими полезными окружениями для красивого отображения сложных формул, а сейчас рассмотрим окружения для создания списков --- нумерованных и ненумерованных. Для нумерованных списков существует окружение \verb"enumerate". Каждый элемент списка должен предваряться командой \verb"\item". Пример:\index{enumerate=нумерованный список,itemize=ненумерованный список,\item}
\begin{enumerate}
\item Вот дом, который построил Джек.
\item А это пшеница, которая в тёмном чулане хранится в доме, который построил Джек.
\item А это синица, которая ловко ворует пшеницу. Синице по мере скудных возможностей мешают:
    \begin{enumerate}
    \item Кот.
    \item Старый пёс без хвоста.
    \item Корова безрогая.
    \end{enumerate}
\end{enumerate}
Как видите, списки вполне могут вкладываться друг в друга.
\par Для ненумерованных списков есть окружение \verb"itemize". Попробуйте его сами в следующем упражнении!


\section{Упражнение}
\begin{staticpart}
Наберите следующий текст: \htmlblockquote{В числе прочего, философия языка программирования Python предполагает соблюдение следующих принципов:
\begin{itemize}
\item Красивое лучше, чем уродливое.
\item Явное лучше, чем неявное.
\item И ещё два связанных друг с другом принципа: \begin{enumerate}
\item Простое лучше, чем сложное.
\item Сложное лучше, чем запутанное.
\end{enumerate}
\end{itemize}}
\end{staticpart}
В числе прочего, философия языка программирования Python предполагает соблюдение следующих принципов:
Красивое лучше, чем уродливое.
Явное лучше, чем неявное.
И ещё два связанных друг с другом принципа: Простое лучше, чем сложное. Сложное лучше, чем запутанное.