\section{Важные рекомендации по работе с уроком (нажмите для просмотра) ←}
\par Урок интерактивный. Весь текст урока набран в системе \LaTeX, причём исходный код отображается параллельно <<чистовому>> варианту. Сейчас Вы должны это видеть во всей красе. Кроме того, при наведении курсора на любую формулу, кроме формул в упражнениях, рядом отображается код формулы. Постоянно следите, как набрана та или иная отображаемая формула в уроке; это важный элемент обучения. Больше того, Вы можете на ходу вносить изменения в код урока и эти изменения будут отображаться по нажатии сочетания \verb"Ctrl+Enter". Так что смело экспериментируйте прямо здесь, на месте. Но помните, что это всё же не полноценный \LaTeX, и некоторые эксперименты нужно будет делать в реальной \LaTeX-среде, например, в \href{https://papeeria.com/landing}{Papeeria} или \href{https://www.overleaf.com/index_b}{Overleaf}. Буду специально оговаривать, когда эксперименты в уроке могут не сработать.
\par Весь урок разбит на совсем небольшие <<шаги>> с объяснениями. Эти шаги разбавлены упражнениями для закрепления материала. Упражнения являются неотъемлемой частью урока и их крайне рекомендуется выполнить все: материал запомнится существенно лучше. Если (и только если!) возникают ощутимые трудности с выполнением упражнения, можно подсмотреть код формулы в упражнении, щёлкнув по формуле мышью.
\par Если Вы уже работали с этим уроком и хотите лишь быстро вспомнить конкретный материал, используйте поиск по ключевым словам вверху страницы. При вводе команды автоматически откроются шаги, где эта команда упомянута и в шагах будут подсвечены формулы, в которых она использована.