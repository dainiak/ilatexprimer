\section{Индексы (и степени)}
\par Наверное, первое, с чего начинается набор формул --- это с индексов.\index{индекс,нижний индекс=_}
Нижний индекс можно организовать так: \(a_i\).
Верхний индекс так: \(e^x\).\index{степень=верхний индекс=^}
Заметьте, что когда индекс содержит более одной буквы, может случиться казус: $a_index$. Чтобы избежать таких недоразумений, группируйте, используя фигурные скобки, вот так: $a_{index}$.
Можно комбинировать верхний и нижний индексы: $a_i^2$ и $a^2_i$ --- посмотрите в коде, как по-разному набраны эти две неотличимые на вид формулы. В целом, порядок ввода индексов --- дело хозяйское, но лучше начинать с того индекса, который является неотъемлемой частью обозначения объекта. Например, если верхний индекс обозначает возведение в степень, лучше набрать \verb"a_i^2".

\section{Греческие буквы}
\par Помимо стандартных латинских букв $a$, $b$, и так далее, в математике часто встречаются греческие буквы. Они набираются с помощью команд, совпадающих с именем букв. При этом команды, соответствующие заглавным греческим буквам, сами начинаются с заглавной буквы:
\[\alpha, \beta, \gamma, \delta, \xi, \Delta, \Omega\]\index{греческие буквы}
и так далее. Список команд для греческих букв приведён по \href{https://www.latex-tutorial.com/symbols/greek-alphabet/}{ссылке}.

\section{Шрифты математического режима}
\par Окей, греческие буквы есть. А как быть, когда нужно набрать латинскую букву, но другим шрифтом? Одна из частых таких ситуаций --- обозначения числовых множеств \(\mathbb{N}, \mathbb{Z}, \mathbb{Q}, \mathbb{R}\)\index{«двойной» жирный математический шрифт=\mathbb}. Или, скажем, прямой шрифт для обозначения вероятности \(\mathrm{P}\) или операторов \(\mathrm{diam}, \mathrm{argmin}\)\index{«прямой» математический шрифт=\mathrm}, и т.\,д. А иногда требуется вот такой <<рукописный>> шрифт: \(\mathcal{A}\)\index{\mathcal=каллиграфический (рукописный) математический шрифт} или ещё более похожий на рукописный \(\mathscr{A}\)\index{\mathscr}. Реже других, но всё же встречается и готический шрифт: \( \mathfrak{B} \)\index{готический шрифт=\mathfrak}. В общем, если Вы посмотрите в код, то и сами всё увидите: для изменения шрифта есть команды \verb"\mathbb", \verb"\mathrm", \verb"\mathcal", \verb"\mathscr" и другие. Полный список приведён, например, \href{https://tex.stackexchange.com/a/58124}{тут}. Казалось бы, уже для одних только шрифтов нужно запоминать кучу команд. Ответ: не нужно, просто распечатайте список шрифтовых команд или сохраните на компьютере в легко доступном месте, и вместо запоминания команд просто помните, где их подсмотреть при необходимости. Необязательно помнить весь арсенал команд даже основных пакетов \LaTeX.

\section{Упражнение}
\begin{staticpart}
Наберите текст:\htmlblockquote{\par Если взять любую пару чисел \(\showSourceOnClick \alpha\) и \(\showSourceOnClick \beta\) из множества рациональных чисел \(\showSourceOnClick \mathbb{Q} \), то число \(\showSourceOnClick (\alpha+\beta)^2\) также будет рациональным. Кстати, во множестве \(\showSourceOnClick \mathbb{R}\) можно найти число \(\showSourceOnClick \gamma\), такое, что \(\showSourceOnClick \gamma^2\) рационально, а само \(\showSourceOnClick \gamma\) --- нет. Подумайте, как бы найти такие \emph{иррациональные} числа \(\showSourceOnClick \alpha_1\) и \(\showSourceOnClick \alpha_2\), для которых число \(\showSourceOnClick \alpha_1^{\alpha_2}\) было бы \emph{рациональным}.}
\end{staticpart}
\par Если взять любую пару чисел ??? и ??? из множества рациональных чисел ???, то число ??? также будет числом рациональным. Кстати, во множестве ??? можно найти число ???, такое, что ??? рационально, а само ??? --- нет. Подумайте, как бы найти такие иррациональные числа ??? и ???, для которых число ??? было бы рациональным.

\section{Корни}
\par В этом шаге посмотрите обязательно на код. Корень квадратный можно набрать так: \( \sqrt{2} \).\index{корень квадратный=\sqrt} Если нужно набрать корень специфичной степени, делается это так: \( \sqrt[3]{2} \). Кстати, можно в первом случае не указывать фигурные скобки и набрать \( \sqrt 2 \), но вот при наборе корня из двузначного числа уже может быть проблема: \( \sqrt 25 \) не работает. Здесь как раз уместно рассказать, как устроены команды в \LaTeX. Команды нужны для набора многих вещей: корней, дробей и прочего. У команды может быть несколько обязательных параметров и несколько (как правило, ноль или один) необязательный параметр. Например, команда \verb"\sqrt" имеет один обязательный параметр: над чем ставить корень. Необязательный аргумент --- какой степени корень. Необязательные аргументы указываются в квадратных скобках.\index{необязательный аргумент команды} Обязательные аргументы либо идут через пробел после имени команды, либо указываются (и одновременно группируются) в фигурных скобках. Чаще всего, группировка оказывается необходима, поэтому я рекомендую явно ставить фигурные скобки вокруг параметров команд.
\par Отдельно нужно сказать, что в математических статьях обычно корни более чем второй степени указываются не значком корня, а именно как степень, например, с точки зрения академического стиля почти всегда уместнее набрать \( 2^{1/3} \) нежели \( \sqrt[3]{2} \).

\section{Дроби и биномиальные коэффициенты}
\par Дроби в \LaTeX набираются с помощью команды \verb"\frac", два обязательных параметра которой --- числитель и знаменатель дроби: \( \frac{a}{b} \). Аналогично устроена команда \verb"\binom", предназначенная для набора биномиальных коэффициентов: \( \binom{n}{k} = \binom{n}{n-k} \).\index{дробь=\frac=\over,биномиальный коэффициент=\binom=\choose}
\par Для полноты картины отмечу, что в \TeX есть инфиксные команды \verb"\over" и \verb"\choose" для отображения дробей и биномиальных коэффициентов, но они не рекомендуются к использованию в \LaTeX.

\section{Скобки}
\par Скобки в \LaTeX играют особенную роль, как Вы убедились на предыдущих шагах: фигурные скобки группируют контент, а в квадратных скобках иногда указываются необязательные параметры команд, как в случае с корнями. На самом деле, квадратные скобки обычно отображаются сами по себе: \( [a,b] \), а вот чтобы вывести фигурные скобки на экран, перед ними придётся обязательно поставить бэкслэш: \( \mathbb{N} = \{1,2,3,\ldots\} \). Без него было бы тоскливо: \( \mathbb{N} = {1,2,3,\ldots} \). А вот с круглыми скобками проблем нет: они всегда отображаются на экране.
\par Особого внимания заслуживает растяжение скобок. Например, пусть хочется отобразить дробь в какой-то степени: \( (\frac{x^4}{a^b})^7 \). Как видно, дробь как бы <<перерастает>> окружающие её скобки, и в дисплейном режиме всё выглядит ещё несуразнее: \[ (\frac{x^4}{a^b})^7 .\]
Чтобы растянуть скобки по вертикали до размера дроби, перед открывающей скобкой ставят команду \verb"\left"\index{модификаторы размера скобок=\left=\right}, а перед закрывающей --- команду \verb"\right". Посмотрите, насколько лучше это выглядит: \[ \left(\frac{x^4}{a^b}\right)^7 .\]
Конечно же, пара команд \verb"\left…\right" может быть использована не только с круглыми, но и с фигурными и квадратными скобками.
\par Не стоит всё же злоупотреблять командами \verb"\left" и \verb"\right" и вставлять их везде --- подробности по \href{https://tex.stackexchange.com/a/58641}{ссылке}. Общее правило такое: вставляйте \verb"\left" и \verb"\right" там, где видите явное несоответствие между высотой скобок и того, что они окружают.
\par Автору-перфекционисту может иногда не хватать команд \verb"\left" и \verb"\right". Например, в формуле \[\left( 1 + \left( \left( a+b \right)/\left(a-b\right) \right) \right)^5\] хотелось бы, чтобы внешние скобки были больше внутренних, но поскольку подформул большой высоты нет, команды \verb"\left" и \verb"\right" тут попросту бесполезны. Пригодятся команды \verb"\bigl", \verb"\Bigl", \verb"\biggl" и \verb"\Biggl"\index{\bigl,\Bigl,\biggl,\Biggl,\bigr,\Bigr,\biggr,\Biggr} и их <<праворучные>> аналоги с буквой r на конце: \[\Bigl( 1 + \bigl( (a+b)/(a-b) \bigr) \Bigr)^5.\]
Также команды big-типа подойдут, когда \verb"\left" и \verb"\right" слишком уж стараются и скобки начинают визуально доминировать над выражением, что они окружают: из двух формул
\[
\left( \sum_{i=1}^{n-1} i \right)
\quad\text{vs.}\quad
\biggl(\sum_{i=1}^{n-1} i\biggr)
\]
вторая обычно смотрится лучше.

\section{Ещё скобки}
\par Помимо обычных житейских круглых, квадратных и фигурных скобок, в математике есть другие парные символы <<скобочного типа>>: вот примеры: \( \lvert x \rvert \) (или просто \( |x| \)), \( \lVert x \rVert \) (можно и так: \( \| x \| \)), \( \lfloor x \rfloor \), \( \lceil x \rceil\). Разумеется, команды \verb"\left" и \verb"\right", описанные в предыдущем шаге, со всеми перечисленными скобочными символами также работают.\index{норма=двойная вертикальная черта=\lVert=\rVert,верхняя целая часть=\lceil=\rceil,нижняя целая часть=\lfloor=\rfloor,абсолютное значение=вертикальная черта=\lvert=\rvert=\|}

\section{Упражнение}
\begin{staticpart}
Наберите формулу: \[\showSourceOnClick \sqrt[4]{\left\lvert\frac{a}{b}+\frac{c}{d}\right\rvert}=\frac{|ad+bc|^{1/4}}{|bd|^{1/4}}\]
\end{staticpart}
\[?\]

\section{Логические символы}
\par Нередки случаи, когда вместо слов мы используем стандартные кванторы существования и всеобщности и стрелки для обозначения логического следования. Для кванторов есть команды \verb"\exists" и \verb"\forall". Для стрелок есть команды с настолько же говорящими (для знающих английский) именами \verb"\implies" и \verb"\iff".\index{знак логического следования=если…то…=\implies,знак логического тождества=тогда и только тогда=\iff}
\par Также могут пригодиться и символы, обозначающие конъюнкцию, дизъюнкцию и отрицание: \verb"\land", \verb"\lor", \verb"\lnot"\index{дизъюнкция=логическое ИЛИ=\lor=\vee,конъюнкция=логическое И=\land=\wedge,логическое отрицание=\lnot=\neg} (буква \verb"l" перед названиями команд для логических операций возникла от слова \emph{logical}). Пример использования всего перечисленного: \[ \exists x \forall y (x\land y=x\lor y \iff \lnot y=\lnot x) \]
\par Ещё два замечания. Во-первых, у команд \verb"\land", \verb"\lor", \verb"\lnot" есть команды-синонимы \verb"\wedge", \verb"\vee", \verb"\neg" --- выбирайте вариант на свой вкус. Во-вторых, командам \verb"\implies" и \verb"\iff", отображающим довольно длинные стрелки с отступами по бокам, альтернативой могут служить команды \verb"\Rightarrow" и \verb"\Leftrightarrow", выводящие стрелки покороче и без отступов.\index{\Rightarrow,\Leftrightarrow}

\section{Стрелки}
\par На предыдущем шаге мы познакомились с <<логическими>> стрелками \verb"\implies" и \verb"\iff" и их <<обычными>> аналогами \verb"\Rightarrow" и \verb"\Leftrightarrow". Неудивительно, что в \LaTeX есть ещё команда \verb"Leftarrow". Заметьте, что три последние упомянутые команды все начинаются с заглавных букв. Команды \verb"\leftarrow", \verb"\rightarrow" и \verb"\leftrightarrow" тоже есть и выводят они обычные, <<тонкие>> стрелки: \(\leftarrow,\rightarrow,\leftrightarrow\). У команды \verb"\rightarrow" есть ещё команда-синоним \verb"\to", введённая для краткости.\index{стрелка влево=\leftarrow,стрелка вправо=\rightarrow=\to,стрелка в обе стороны=\leftrightarrow}

\par В \LaTeX стрелки очень гибки, и можно делать такие вещи: \( f\xrightarrow{x\rightarrow 7}\infty \). Как видно, у команды \verb"\xrightarrow"\index{стрелка вправо с подписью=\xrightarrow} один обязательный аргумент --- это то, что должно быть отображено над стрелкой. Растянется она по горизонтали автоматически, чтобы вместить всё необходимое. Кстати, у \verb"\xrightarrow" есть ещё и \emph{необязательный} аргумент --- то, что может быть дополнительно изображено \emph{под} стрелкой. Попробуйте самостоятельно его в действии! А ещё есть команда \verb"\xleftarrow" --- тут, думаю, комментарии излишни. Заметьте, что по ходу дела мы и с командой для значка бесконечности познакомились!
\par В \LaTeX имеются ещё универсальные команды \verb"\overset" и \verb"\underset", которые размещают любые символы соответвенно над и под заданным значком. Например: \( a \overset{\Delta}{=} b \) и \( a \underset{c}{=} b \).\index{разместить сверху от значка=\overset,разместить снизу от значка=\underset} Если нужно что-разместить с обеих сторон значка, это тоже выполнимо: \( a \underset{!}{\overset{?}{=}} b \).
Следующее --- что стрелки набираются командами \verb"\rightarrow", \verb"\Rightarrow" и другими (угадайте, как набрать стрелки, смотрящие влево;)). Есть синоним для команды \verb"\rightarrow" --- это команда \verb"\to", введённая для удобства: набирать её быстрее.

\par Чтобы поставить стрелку над выражением, можно воспользоваться командой \verb"\overrightarrow"\index{\overrightarrow=стрелка над выражением}: \(\overrightarrow{abc}\). Думаю, Вы сами догадаетесь, что, скажем, делает команда \verb"\underleftarrow";)\index{\overleftarrow,\underleftarrow=стрелка под выражением,\underrightarrow=стрелка под выражением}

\section{Упражнение}
\begin{staticpart}
Наберите формулы: \[\showSourceOnClick  \forall x\forall y(\lnot (x\land y)=\lnot x\lor \lnot y) \] \[\showSourceOnClick  x^2 = y^2 \iff x=y \lor x+y=0 \]
\end{staticpart}
\[ ? \]
\[ ? \]

\section{Как набрать формулировку из матана}
\par Вряд ли можно набрать много математического текста и не столкнуться со значками бесконечности, кванторами и стрелками. Вот примеры, которые знакомят Вас со всеми этими вещами:
\[\lim_{x\rightarrow x_0}f(x)=f_0 \iff ( \forall \epsilon\in\mathbb{R}^+ \exists \delta\in\mathbb{R}^+: |x-x_0|<\epsilon \implies |f-f_0|<\delta ). \]
\par Из этого примера мы узнаём, что значок принадлежности элемента множеству набирается командой \verb"\in". Ещё в разобранном примере мы знакомимся с командой \verb"\lim"\index{\lim}, которая ведёт себя интересным образом: то, что после неё мы набираем как бы в нижнем индексе, заносится вообще под слово lim, как мы и привыкли видеть в книжках. Но это происходит только в display-режиме. В inline-режиме \LaTeX экономит вертикальное место и команда \verb"\lim" даст такой результат: \( \lim_{x\rightarrow x_0}f(x) \).
\par Вы могли обратить внимание ещё на пару вещей: во-первых, букву \( \epsilon \) мы привыкли видеть в учебниках другой, а именно, такой: \( \varepsilon \). Во-вторых, отсутствие отступа перед квантором существования и не слишком большой отступ после двоеточия несколько ухудшают читабельность формулы. Давайте добавим отступы и привычный эпсилон, а команду \verb"\rightarrow" заменим на короткий синоним \verb"\to":
\[\lim_{x\to x_0}f(x)=f_0\quad \iff\quad ( \forall \varepsilon\in\mathbb{R}^+ \, \exists \delta\in\mathbb{R}^+:\: |x-x_0|<\varepsilon \implies |f-f_0|<\delta ). \]
Отступы мы создали командами \verb"\quad" (вставляет учетверённый пробел), \verb"\," (вставляет узкий пробел) и \verb"\:" (вставляет пробел, подходящий после двоеточия).\index{\quad=четверной пробел,узкий пробел,\:}

\section{Математические операторы}
\par Математические операторы, являющиеся сокращениями от латинских слов, принято набирать прямым шрифтом. Чаще всего, в \LaTeX находится команда, которая это делает, называющаяся так же, как и сам оператор: \verb"\lim", \verb"\sup", \verb"\min", и так далее.\index{математические операторы,\lim=предел,\sup=супремум,\inf=инфимум,\min=минимум,\max=максимум} Однако для некоторых операторов соответствующая команда не находится. В этом случае используйте команду \verb"\operatorname".\index{\operatorname} Вот пример: \[3=\operatorname{argmin}_{x\in\mathbb{N}}(x^2-8).\] У команды \verb"\operatorname" есть сестра --- команда \verb"\operatorname*" (да-да, в \LaTeX звёздочка может быть полноценным символом в имени команды), которая отличается тем, как будут отображаться «пределы» у оператора в режиме display-math. Поставьте сами звёздочку в коде в примере выше, нажмите \verb"Ctrl+Enter" и увидите, что получится!
\par Когда нужно набрать прямым шрифтом лишь одну букву, не являющуюся по смыслу оператором, можно использовать рассмотренную нами ранее чисто шрифтовую команду \verb"\mathrm".\index{\mathrm}
\par Было бы странно не упомянуть об операторах типа суммирования, вот они:
\[\sum_{k=0}^{100} k^2\]\index{\sum=сумма}
\[\prod_{k=1}^{100} k\]\index{\prod=произведение}
\[\int_{-\pi}^\pi\sin x\,\mathrm{d}x.\]\index{\int=интеграл}
Можете также поэкспериментировать с операторами \verb"\bigcap", \verb"\bigcup", \verb"\bigsqcup", важными для теоретико-множественных формул.\index{пересечение множеств=\bigcap,объединение множеств=\bigcup,дизъюнктное объединение множеств=\bigsqcup}

\section{Упражнение}
\begin{staticpart}
Наберите текст: \htmlblockquote{\par Существует конечный предел \[\showSourceOnClick \lim_{n\to\infty}\left(-\ln n + \sum_{k=1}^n \frac{1}{k}\right),\] называемый \emph{постоянной Эйлера--Маскерони} и заключённый в диапазоне \(\showSourceOnClick [0.5,\,0.6]\).}
\end{staticpart}
\par Существует конечный предел \[?\] называемый постоянной Эйлера--Маскерони и заключённый в диапазоне \(?\).

\section{Диакритические знаки}
\par «Икс с чертой», «игрек с крышкой», «зет штрих», «вектор тау» --- всё возможно: \(\bar{x},\, \hat{y},\, z',\, \vec{\tau}\).\index{диакритические знаки,штрих у символа,крышка над символом=\hat,черта над символом=надчёркивание=\bar=\overline,тильда над символом=волна над символом=\tilde=\widetilde} Полный список стандартных команд для добавления символам «головных уборов» можно посмотреть \href{https://en.wikibooks.org/wiki/LaTeX/Special_Characters#Math_mode}{по ссылке}.
\par Знать ещё обязательно нужно, пожалуй, только следующее. Когда нужно «навесить крышу» на длинное выражение, нужно использовать <<длинные>> варианты команд:\index{\hat=\widehat,\vec=\overrightarrow,\tilde=\widetilde}
\[\tilde{abcd} \text{ vs. } \widetilde{abcd}\]
\[\hat{abcd} \text{ vs. } \widehat{abcd}\]
\[\bar{abcd} \text{ vs. } \overline{abcd}\]
\[\vec{abcd} \text{ vs. } \overrightarrow{abcd}\]
\par Обратите внимание: для того, чтобы включить в тело формулы сопроводительный текст ``vs.'', не несущий непосредственной математической нагрузки, используется команда \verb"\text".\index{размещение текста внутри формулы=\text}

\section{Подчёркивание и надчёркивание}
\par Учимся на примере: \[\overline{abc}\quad \underline{abc}\quad \underbrace{abc}_{\text{пояснение}} \quad \overbrace{abc}^{\text{пояснение}}\]\index{черта под символом=подчёркивание=\underline,черта над символом=надчёркивание=\overline,фигурная скобка над выражением=\overbrace,фигурная скобка под выражением=\underbrace}
Заметьте, что когда внутри формулы нужно дать текстовые пояснения, их заключают в команду \verb"\text"\index{размещение текста внутри формулы=\text}, которая выводит то, что ей передают, в текстовом, а не математическом режиме. Внутри команды \verb"\text" можно временно перейти в математический режим:
\[ \underbrace{a^3=abc}_{\text{потому что \(a^2=bc\)}} \]

\section{Упражнение}
\begin{staticpart}
Наберите текст: \htmlblockquote{\par Число, записываемое в двоичной системе так: \[\showSourceOnClick  \underbrace{11\ldots 1}_{\text{$k$ единиц}},\] равняется \(2^k-1\). По определению мы полагаем \(\showSourceOnClick 2^k\overset{\mathrm{def}}{=} \overbrace{2\cdot 2\cdot\ldots \cdot 2}^{\text{$k$ раз}}\).}
\end{staticpart}
\par Число, записываемое в двоичной системе так: \[11\ldots 1,\] равняется \(2^k-1\).
По определению мы полагаем \(2^k=2 \cdot 2 \cdot\ldots\cdot 2\).

\section{Неравенства и \(O\)-символика}
\par Равенство и строгие неравенства в \LaTeX обозначаются теми значками, которые непосредственно есть на клавиатуре. Для нестрогих неравенств есть команды \verb"\le" и \verb"\ge". Но более привычны для, например, российской типографики, значки, выводимые командами \verb"\leqslant" и \verb"\geqslant". Сравните: \(\le,\ge,\leqslant,\geqslant\).

\section{Вычисления по модулю}
\par В системе \LaTeX есть целых три команды, которые выводят слово \emph{mod} в разных вариациях. Проще всего рассмотреть пример:\index{mod=\mod=\pmod=\bmod=по модулю=остаток от деления}
\begin{itemize}
\item Рассмотрим число \(c\), равное \(a\bmod b\).
\item Имеет место равенство: \[a=c\pmod{b}\]
\item Имеет место равенство: \[a=c\mod b\]
\end{itemize}