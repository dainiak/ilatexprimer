\section{\(\LaTeX\) is a programming/markup language}
\index{LaTeX}
\par \LaTeX is basically a markup language. Another example of markup language that you surely encounter daily is HTML, enabling practically any site on the Internet, but \LaTeX is far more powerfull (as it supports, e.g., macro-programming). When preparing a document using \LaTeX, you write a source file, and a special \LaTeX-compiler creates a neat PDF-document using your source (some years ago the standard output for \LaTeX was not PDF, but rather DVI or PS, but these two formats are getting obsolete for ordinary computer users).

\par \LaTeX is based on \TeX, a markup language designed by a legend of computer science \href{https://en.wikipedia.org/wiki/Donald_Knuth}{Donald Knuth}. In maths and sciences these two languages eventually became the de-facto standard for writing research papers and conference slides, as they are practically unrivalled when it comes to power, tunability and portability. Even modern versions of Microsoft Word and Microsoft PowerPoint (starting with 2007) support inserting formulas using \LaTeX syntax. On the internet you may have heard of \href{https://en.wikipedia.org/wiki/MathJax}{MathJax}\index{MathJax} system that processed \LaTeX code within any web page on-the-fly to display math on discussion boards and elsewhere on the Internet. This system also powers the current lesson.

\par As \LaTeX is not a \href{https://en.wikipedia.org/wiki/WYSIWYG}{WYSIWYG-system}, there is one obvious obstacle: to efficiently use \LaTeX, you need first to devote some time to learning it. Yet to my experience, to be able to encode most of everyday math and to prepare some not-too-complex documents, one suffices to spend about 2-3 hours learning the basics, and then just keep a printed \href{https://wch.github.io/latexsheet/}{cheat-sheet} containing the most common features. While using \LaTeX in practice, you will then learn anything necessary for your work on the way without interrupting for special training.


\section{Literal reproduction}
\par Sometimes you may need to display a text without processing it with \LaTeX . In this case you will need \verb"\verb"\index{\verb} command, followed by the text. It is very useful, assuming that by default \LaTeX would otherwise make many substitutions to the text: for example, doubled “lower-than” symbols \verb"<<…>>"\index{quotes} would be converted into <<quotes>>, and \verb"---"\index{dash} --- into a dash. How would one then display quotes within \verb"\verb"? The answer is that you just need to use a different symbol for delimiting the content of verbatim output, e.g. \verb+\verb|quotation symbol: "|+. (Look at the source code to the right now!)\index{\verb=literal reproduction=verbatim}
\par One should take special care with the following symbols: \verb"& { } $ % # _". For instance, the symbol \%, without preceding backslash, is considered as a start of a comment, which would be skipped during the  compilation, just like in any programming language.\index{comment}%See, for instance, this comment that was omitted in the output on the left.


\section{Paragraphs}
\par It is a good practice to thoughtfully split the text into paragraphs. A new paragraph can be started with a command \verb"\par"\index{\par} or when \LaTeX encounters two consequitive newline characters in the text. The latter is not supported in the current lesson, though.\index{\par=start a new paragraph}


\section{Emphasizing parts of the text}
\par To visually mark some portion of a text, we can use \verb"\textbf"\index{\textbf=bold font} (\textbf{bold} font) and \verb"\textit"\index{\textit=italic font} (\textit{italic} font), and you can \textbf{com\textit{bin}e them}. There is also a text underlining command \verb"\underline", but the majority of modern typesetting textbooks recomment to avoid underlining text, and this command is not supported within the current lesson;)

\par One more command worth mentioning is \verb"\emph"\index{\emph=emphasizing text (italic)}, that tries to make the text stand out automatically, taking into account the context: e.g. if the marked portion lies within italicised paragraph, the portion itself will be typeset using ordinary font. So by default it is recommended to use \verb"\emph" to emphasize portions of the text. Beware: in this web lesson the \verb"\emph" command is far less sophisticated than in real \LaTeX.

\par At times one needs to put an emphasis on a letter. To do this, \LaTeX supports diacritics over the letters like this: \verb"\'{o}"\index{emphasis,diacritics,diacritic marks} (one can use almost \'{a}ny l\'{e}ter under diacritic mark). More on that \href{https://en.wikibooks.org/wiki/LaTeX/Special_Characters#Escaped_codes}{here}.

\par So far we’ve covered bare minimum of text formatting. But what \LaTeX really shines at is math typesetting. Proceed to learn this beauty.


\section{Two main modes: text mode and math mode}
\par Although \LaTeX is surely suitable for typesetting books containing no math at all, the essence of \LaTeX lies in scientific texts. The first thing one should mind about \LaTeX is the presence of two distinct modes: \textbf{normal (ordinary text)} mode and \textbf{math (formulae)} mode. Usually whet typesetting a paper one knows what symbols have standard language meaning and what symbols belong to mathematical notation. For \LaTeX to get this distinction we should enclose pieces of text having mathematical meaning in dollars or in \verb"\(" \verb"\)" delimiters: e.g. $F=ma$ is a formula enclosed in dollars, so \LaTeX recognized it and used special font and formatting rules to display it, whereas F=ma is the same formula which was forgotten to be enclosed in dollars and so it is output using standard text font and formatting, making it look awkward compared to the first one. Instead of using \verb"$…$" one can use \verb"\(…\)" bringing the same result: \(F=ma.\) It is recommended (though not strongly) to use the \verb"\(…\)" pair of delimiters instead of \verb"$…$", as in the former pair it is obvious which delimiter is the “opening” and which is the “closing”, whereas for \verb"$…$" pair \LaTeX has to make a guess about it.


\section{Two math output modes: inline and display}
\par As in well-typeset papers and books, \LaTeX can both place a formula inside the text or putting it into attention by placing it on a separate line (such formulas are called \emph{display} formulas). When a formula is located in a text environment, this mathematical mode is called \emph{inline math}, and when the formula is on a separate line -- a \emph{display math}. To put the formula in the middle of a single line, it is enclosed in double dollars: \verb"$$...$$" in the \TeX system and in the square ``backslash'' parentheses \verb"[...]", -- in \LaTeX. Here’s an example: \[F=ma.\]
Unlike the purely virtual difference between \verb"$...$" and \verb"\(...\)" for inline math, the difference between \verb"$...$" and \verb"\[...\]" is important: although sometimes the difference is not visible, there may well be a situation when the formula, being surrounded with double dollars, has spacing with respect to from the surrounding text that is not intended.\index{display math}

\section{Exercise: text formatting}
\begin{staticpart}
Typeset the following text, keeping the paragraphs, font face, and the types of formulas (diaplay or inline). For your convenience, the unformatted text is already pasted in the input field:
\htmlblockquote{\par It is really easy to typeset any text that contains \textbf{bold font}, and sometimes even \textit{italicised font}, although it is generally better to emphasize portions of the text with \verb"\emph" command.
\par Splitting text into paragraphs and typesetting ``nice quotes'' --- that is easy! But the greatest thing is the typesetting of formulas, as you can communicate beautifully to the world that \(\showSourceOnClick A+B=B+A\) or even shout out that \[\showSourceOnClick C+D=D+C.\]}
\end{staticpart}
It is really easy to typeset any text that contains bold font, and sometimes even italicised font, although it is generally better to emphasize portions of the text with emph command. Splitting text into paragraphs and typesetting nice quotes --- that is easy! But the greatest thing is the typesetting of formulas, as you can communicate beautifully to the world that A+B=B+A or even shout out that C+D=D+C.


\section{Lists}
\par Besides the commands in \LaTeX there are also  \emph{environments}.\index{environment} Environments should be created with \verb"\begin{…}" and \verb"\end{…}". We will later learn some useful environments for beautiful typeset of complex formulas, and right now we will consider a case for using environments in ordinary text typesetting --- to typeset numbered and unnumbered lists. To create a numbered list we use \verb"enumerate" environment. Each element of the list must start with \verb"\item". Here is an example:\index{enumerate=numbered list=ordered list,itemize=unnumbered list=unordered list,\item}
\begin{enumerate}
\item Mercury,
\item Venus,
\item Earth,
\item Mars,
    \begin{enumerate}
    \item Phobos,
    \item Deimos.
    \end{enumerate}
\end{enumerate}
As you can see, the lists can be nested.
\par To typeset unnumbered lists we use \verb"itemize" environment. You will have chance to use it in the next exercise.


\section{Exercise: lists}
\begin{staticpart}
Typeset the following: \htmlblockquote{Among others, the so-called ``Zen of Python'' by Tim Peters lists the following aphorisms:
\begin{itemize}
\item Beautiful is better than ugly.
\item Explicit is better than implicit.
\item And two more somewhat related principles: \begin{enumerate}
\item Simple is better than complex.
\item Complex is better than complicated.
\end{enumerate}
\end{itemize}}
\end{staticpart}

Among others, the so-called Zen of Python by Tim Peters lists the following aphorisms: Beautiful is better than ugly. Explicit is better than implicit. And two more somewhat related principles: Simple is better than complex. Complex is better than complicated.