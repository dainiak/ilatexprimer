\section{\(\LaTeX\) is a programming/markup language}
\index{LaTeX}
\par \LaTeX is basically a markup language. Another example of markup language that you surely encounter daily is HTML, enabling practically any site on the Internet, but \LaTeX is far more powerfull (as it supports, e.g., macro-programming). When preparing a document using \LaTeX, you write a source file, and a special \LaTeX-compiler creates a neat PDF-document using your source (some years ago the standard output for \LaTeX was not PDF, but rather DVI or PS, but these two formats are getting obsolete for ordinary computer users).

\par \LaTeX is based on \TeX, a markup language designed by a legend of computer science \href{https://en.wikipedia.org/wiki/Donald_Knuth}{Donald Knuth}. In maths and sciences these two languages eventually became the de-facto standard for writing research papers and conference slides, as they are practically unrivalled when it comes to power, tunability and portability. Even modern versions of Microsoft Word and Microsoft PowerPoint (starting with 2007) support inserting formulas using \LaTeX syntax. On the internet you may have heard of \href{https://en.wikipedia.org/wiki/MathJax}{MathJax}\index{MathJax} system that processed \LaTeX code within any web page on-the-fly to display math on discussion boards and elsewhere on the Internet. This system also powers the current lesson.

\par As \LaTeX is not a \href{https://en.wikipedia.org/wiki/WYSIWYG}{WYSIWYG-system}, there is one obvious obstacle: to efficiently use \LaTeX, you need first to devote some time to learning it. Yet to my experience, to be able to encode most of everyday math and to prepare some not-too-complex documents, one suffices to spend about 2-3 hours learning the basics, and then just keep a printed \href{https://wch.github.io/latexsheet/}{cheat-sheet} containing the most common features. While using \LaTeX in practice, you will then learn anything necessary for your work on the way without interrupting for special training.


\section{Literal reproduction}
\par Sometimes you may need to display a text without processing it with \LaTeX . In this case you will need \verb"\verb"\index{\verb} command, followed by the text. It is very useful, assuming that by default \LaTeX would otherwise make many substitutions to the text: for example, doubled “lower-than” symbols \verb"<<…>>"\index{quotes} would be converted into <<quotes>>, and \verb"---"\index{dash} --- into a dash. How would one then display quotes within \verb"\verb"? The answer is that you just need to use a different symbol for delimiting the content of verbatim output, e.g. \verb+\verb|quotation symbol: "|+. (Look at the source code to the right now!)\index{\verb=literal reproduction=verbatim}
\par One should take special care with the following symbols: \verb"& { } $ % # _". For instance, the symbol \%, without preceding backslash, is considered as a start of a comment, which would be skipped during the  compilation, just like in any programming language.\index{comment}%See, for instance, this comment that was omitted in the output on the left.


\section{Paragraphs}
\par It is a good practice to thoughtfully split the text into paragraphs. A new paragraph can be started with a command \verb"\par"\index{\par} or when \LaTeX encounters two consequitive newline characters in the text. The latter is not supported in the current lesson, though.\index{\par=start a new paragraph}


\section{Emphasizing parts of the text}
\par To visually mark some portion of a text, we can use \verb"\textbf"\index{\textbf=bold font} (\textbf{bold} font) and \verb"\textit"\index{\textit=italic font} (\textit{italic} font), and you can \textbf{com\textit{bin}e them}. There is also a text underlining command \verb"\underline", but the majority of modern typesetting textbooks recomment to avoid underlining text, and this command is not supported within the current lesson;)

\par One more command worth mentioning is \verb"\emph"\index{\emph=emphasizing text (italic)}, that tries to make the text stand out automatically, taking into account the context: e.g. if the marked portion lies within italicised paragraph, the portion itself will be typeset using ordinary font. So by default it is recommended to use \verb"\emph" to emphasize portions of the text. Beware: in this web lesson the \verb"\emph" command is far less sophisticated than in real \LaTeX.

\par At times one needs to put an emphasis on a letter. To do this, \LaTeX supports diacritics over the letters like this: \verb"\'{o}"\index{emphasis,diacritics,diacritic marks} (one can use almost \'{a}ny l\'{e}ter under diacritic mark). More on that \href{https://en.wikibooks.org/wiki/LaTeX/Special_Characters#Escaped_codes}{here}.

\par So far we’ve covered bare minimum of text formatting. But what \LaTeX really shines at is math typesetting. Proceed to learn this beauty.


\section{Two main modes: text mode and math mode}
\par Although \LaTeX is surely suitable for typesetting books containing no math at all, the essense of \LaTeX lies in scientific texts. The first thing one should mind about \LaTeX is the presense of two distinct modes: \textbf{normal (ordinary text)} mode and \textbf{math (formulae)} mode. Usually whet typesetting a paper one knows what symbols have standard language meaning and what symbols belong to mathematical notation. For \LaTeX to get this distinction we should enclose pieces of text having mathematical meaning in dollars or in \verb"\(" \verb"\)" delimiters: e.g. $F=ma$ is a fomula enclosed in dollars, so \LaTeX recognized it and used special font and formatting rules to display it, whereas F=ma is the same formula which was forgotten to be enclosed in dollars and so it is output using standard text font and formatting, making it look awkward compared to the first one. Instead of using \verb"$…$" one can use \verb"\(…\)" bringing the same result: \(F=ma.\) It is recommended (though not strongly) to use the \verb"\(…\)" pair of delimiters instead of \verb"$…$", as in the former pair it is obvious which delimiter is the “opening” and which is the “closing”, whereas for \verb"$…$" pair \LaTeX has to make a guess about it.


\section{Two math output modes: inline and display}
\par Как и в хорошо набранных книгах, в \LaTeX можно как располагать формулы внутри текста, так и делать их центром внимания, вынося на отдельную строчку (такие формулы называются \textit{выключными} или \textit{выносными}). Когда формула располагается в окружении текста, такой математический режим называется inline math, а когда формула располагается на отдельной строчке --- display math. Чтобы формула была вынесена на середину отдельной строки, её заключают в двойные доллары: \verb"$$…$$" в системе $\TeX$ и в квадратные <<бэкслэшнутые>> скобки \verb"\[…\]", --- в \LaTeX. Вот пример: \[F=ma.\]
В отличие от чисто виртуального различия между записью \verb"$…$" и \verb"\(…\)" для inline math, разница записей \verb"$$…$$" и \verb"\[…\]" существенна: хотя часто разницы не видно, вполне может возникнуть ситуация, когда формула, оформленная двойными долларами, имеет отступы от окружающего текста не такие, как задумывалось.\index{дисплейный режим=display math}


\section{Exercise}
\begin{staticpart}
Наберите следующий текст, сохраняя разбиение на абзацы, шрифт текста и положение формул (для удобства неоформленный текст уже дан Вам в окне кода):
\htmlblockquote{\par Совсем нетрудно набирать тексты, содержащие \textbf{жирный шрифт}, а иногда и \textit{курсив}, хотя выделять текст лучше всё же командой \verb"\emph".
\par Разбивать текст на параграфы, ставить <<кавычки-ёлочки>> --- дело ТеХники! Но вот что лучше всего, так это набор формул. Ведь так приятно написать, что \(\showSourceOnClick A+B=B+A\) или даже заявить всему миру, что \[\showSourceOnClick C+D=D+C.\]}
\end{staticpart}
Совсем нетрудно набирать тексты, содержащие жирный шрифт, а иногда и курсив, хотя выделять текст лучше всё же командой emph. Разбивать текст на параграфы, ставить кавычки-ёлочки дело ТеХники! Но вот что лучше всего, так это набор формул. Ведь так приятно написать, что A+B=B+A или даже заявить всему миру, что C+D=D+C.


\section{Lists}
\par Помимо команд в \LaTeX есть ещё понятие \emph{окружений}.\index{окружение} Окружения задаются с помощью пары команд вида \verb"\begin{…}" и \verb"\end{…}". Позже мы познакомимся со многими полезными окружениями для красивого отображения сложных формул, а сейчас рассмотрим окружения для создания списков --- нумерованных и ненумерованных. Для нумерованных списков существует окружение \verb"enumerate". Каждый элемент списка должен предваряться командой \verb"\item". Пример:\index{enumerate=нумерованный список,itemize=ненумерованный список,\item}
\begin{enumerate}
\item Вот дом, который построил Джек.
\item А это пшеница, которая в тёмном чулане хранится в доме, который построил Джек.
\item А это синица, которая ловко ворует пшеницу. Синице по мере скудных возможностей мешают:
    \begin{enumerate}
    \item Кот.
    \item Старый пёс без хвоста.
    \item Корова безрогая.
    \end{enumerate}
\end{enumerate}
Как видите, списки вполне могут вкладываться друг в друга.
\par Для ненумерованных списков есть окружение \verb"itemize". Попробуйте его сами в следующем упражнении!


\section{Exercise}
\begin{staticpart}
Наберите следующий текст: \htmlblockquote{В числе прочего, философия языка программирования Python предполагает соблюдение следующих принципов:
\begin{itemize}
\item Красивое лучше, чем уродливое.
\item Явное лучше, чем неявное.
\item И ещё два связанных друг с другом принципа: \begin{enumerate}
\item Простое лучше, чем сложное.
\item Сложное лучше, чем запутанное.
\end{enumerate}
\end{itemize}}
\end{staticpart}
В числе прочего, философия языка программирования Python предполагает соблюдение следующих принципов:
Красивое лучше, чем уродливое.
Явное лучше, чем неявное.
И ещё два связанных друг с другом принципа: Простое лучше, чем сложное. Сложное лучше, чем запутанное.