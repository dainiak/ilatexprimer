\section{Indices and powers (subscripts and superscripts)}
\par Perhaps, the first thing we should learn to typeset are the indices.\index{lower index=subscript=_}
Subscript index can be typeset as follows: \(a_i\).
And here is the superscript: \(e^x\).\index{upper index=power=superscript=^}
Note that when there is more than one symbol in the index, it cannot be just left as is, or we get: $a_index$. To make this typeset properly, we need to group it with curly braces: $a_{index}$.
One can combine subscript and superscript: $a_i^2$ or $a^2_i$ --- if you look at the code you will see that the typeset result does not depend on the order in the source code. Of course, it’s up to you to choose what comes first, yet I recommend to start with the index that is an essential part of the object notation. For instance, if the upper index designates the power, it is recommended to write \verb"a_i^2".

\section{Greek letters}
\par Besides latin letters $a$, $b$, etc, there is a lot of use for the Greek letters in mathematics. They are all typeset using the commands which coincide with the letter names. Also, the command to typeset capital Greek letters start with capital letters themselves:
\[\alpha, \beta, \gamma, \delta, \xi, \Delta, \Omega\]\index{Greek letters=Greek alphabet}
and so on. The list of all the commands for Greek letter typesetting can be found \href{https://www.latex-tutorial.com/symbols/greek-alphabet/}{here}.

\section{Math mode fonts}
\par OK, now that we’ve learned how to typeset Greek letters, what do we do if we want to typeset a latin letter, but with some special font? One of the common cases is the typesetting of the number sets \(\mathbb{N}, \mathbb{Z}, \mathbb{Q}, \mathbb{R}\)\index{blackboard bold mathematical font=\mathbb}. Or, for instance, roman “non-italic” font for probability measure operator \(\mathrm{P}\) or for operators \(\mathrm{diam}, \mathrm{argmin}\)\index{math roman font=math straight font=math non-italic font=\mathrm}, etc. Sometimes we need the following “calligraphic”, “hand-written” letters: \(\mathcal{A}\)\index{\mathcal=calligraphic mathematical font=hand-written mathematical font} or, perhaps, an even more resembling the hand-writing \(\mathscr{A}\)\index{\mathscr=calligraphic mathematical font=hand-written mathematical font}. Somewhat more rare, but still used, is the gothic font: \( \mathfrak{B} \)\index{\mathfrak=gothic font}. Well, if you have looked into the \LaTeX-code of the current section, you now know how to make \LaTeX typeset that: with commands \verb"\mathbb", \verb"\mathrm", \verb"\mathcal", \verb"\mathscr" and others. A more comprehensive list of font commands can be found \href{http://tex.stackexchange.com/a/58124}{here}. Oh, boy, that’s a lot of commands just for setting the font! But don’t worry, it is all memorized when actually being used. For now, I recommend having the printed list of commands that you believe to be useful to you in front of you when you typeset your first texts. In a month of actively using \LaTeX you are likely to no longer need this list;)

\section{Exercise}
\begin{staticpart}
Typeset the following:\htmlblockquote{\par If one takes two rational numbers \(\showSourceOnClick \alpha\) and \(\showSourceOnClick \beta\) from the set of rationals \(\showSourceOnClick \mathbb{Q} \), then the number \(\showSourceOnClick (\alpha+\beta)^2\) is also going to be rational. By the way, one can find a number in \(\showSourceOnClick \mathbb{R}\), say \(\showSourceOnClick \gamma\), such that \(\showSourceOnClick \gamma^2\) is rational, but the \(\showSourceOnClick \gamma\) is not. Think of how to find two \emph{irrational} numbers \(\showSourceOnClick \alpha_1\) and \(\showSourceOnClick \alpha_2\), such that the number \(\showSourceOnClick \alpha_1^{\alpha_2}\) is \emph{rational}.}
\end{staticpart}
\par If one takes two rational numbers ??? and ??? from the set of rationals ???, then the number ??? is also going to be rational. By the way, one can find a number in ???, say ???, such that ??? is rational, but the ??? is not. Think of how to find two irrational numbers ??? and ???, such that the number ??? is rational.

\section{Roots}
\par In this step I strongly suggest you to look at the source code alongside the text. The square root symbol can be typeset as follows: \( \sqrt{2} \).\index{square root=\sqrt} If you need cubic or any other root, employ the optional parameter of \verb"\sqrt": \( \sqrt[3]{2} \). By the way, in the first case you could omit the braces and still get the same result: \( \sqrt 2 \), but when you have a root of some multi-digit number, there might be a problem with the latter way: \( \sqrt 25 \) does not go. This is a fine moment to introduce the way in which \LaTeX commands work. The commands are needed to typeset almost anything beyond single numbers and letters: roots, fractions, and many others. A command may have several required arguments and several (usually zero or one) optional arguments. For instance, the command \verb"\sqrt" has one required argument: what to put under the root symbol. The optional argument for this command is the degree of the root. Optional arguments are passed in the square brackets.\index{optional arguments of commands} Required arguments either go after the name of the command (after an additional space in this case) or are provided (and grouped at the same time) in braces. Most of the time, explicit grouping is needed, so I suggest \emph{always} keeping the braces present around the required arguments.
\par It should be noted separately that in mathematical articles the roots of degree more than 2 are typically typeset as powers. For instance, in professional mathematical texts it is much more common to see \( 2^{1/3} \) than \( \sqrt[3]{2} \).

\section{Fractions and binomial coefficients}
\par In \LaTeX fractions are typeset using \verb"\frac" command with two required arguments --- the numerator and the denominator of the fraction: \( \frac{a}{b} \). The command \verb"\binom" which is used to typeset the symbol of binomial coefficient behaves analogously: \( \binom{n}{k} = \binom{n}{n-k} \).\index{fraction=\frac=\over,binomial coefficient=\binom=\choose}
\par For completeness I need to note that there are also \TeX commands \verb"\over" and \verb"\choose" that are employed in the infix style to typeset fractions and binomial coefficients, but the usage of these commands is not recommended in \LaTeX.

\section{Delimiters: parentheses, brackets, braces, etc.}
\par Delimiters of different types play quite an important role in \LaTeX as you have had chance to see in the previous steps: curly braces group content, and square brackets are used to provide optional arguments to the commands. In fact, square brackets are usually displayed normally: \( [a,b] \), but to display curly braces one has to prepend them with a backslash symbol: \( \mathbb{N} = \{1,2,3,\ldots\} \). Without that we would get: \( \mathbb{N} = {1,2,3,\ldots} \). As for the ordinary parentheses, they are always displayed normally and do not play any special role in \LaTeX coding.
\par One important thing to learn is stretching delimiters. For instance, suppose we need to raise a fraction into some power: \( (\frac{x^4}{a^b})^7 \). As you can see, the fraction ``outgrows'' the parentheses that surround it, and in display math mode the dissonance is even more prominent: \[ (\frac{x^4}{a^b})^7 .\]
To stretch the parentheses for them to suite the fraction height, we need to put a special command \verb"\left"\index{stretching delimiters=\left=\right} before the opening parenthesis and \verb"\right" before the closing one. Just look at how much better it is now: \[ \left(\frac{x^4}{a^b}\right)^7 .\]
Of course, the pair \verb"\left…\right" can be used with brackets and braces in the same manner.
\par As gorgeous as they are, \verb"\left" and \verb"\right" should not be used everywhere, see the details \href{http://tex.stackexchange.com/a/58641}{here}. The rule of thumb is as follows: plug in \verb"\left" and \verb"\right" when you see prominent difference between the delimiters and the height of the expression which is surrounded by them.
\par A perfectionist author may not be happy with just \verb"\left" and \verb"\right". For instance, in \[\left( 1 + \left( \left( a+b \right)/\left(a-b\right) \right) \right)^5\] it would be great if the outermost parentheses were larger than the inner ones, but as the height of all the expressions is the same, the \verb"\left" and \verb"\right" commands are simply useless in this case, as they have no reason to alter the default delimiter height. So we have a reason to employ the new commands \verb"\bigl", \verb"\Bigl", \verb"\biggl" and \verb"\Biggl"\index{\bigl,\Bigl,\biggl,\Biggl,\bigr,\Bigr,\biggr,\Biggr} and their ``right-handed counterparts'' with letter r at the end: \[\Bigl( 1 + \bigl( (a+b)/(a-b) \bigr) \Bigr)^5.\]
These ``big-commands'' are also useful when \verb"\left" and \verb"\right" are overreacting and the parentheses become the dominant symbols in the whole expression: of the two formulas
\[
\left( \sum_{i=1}^{n-1} i \right)
\quad\text{vs.}\quad
\biggl(\sum_{i=1}^{n-1} i\biggr)
\]
the second one looks better.

\section{More on delimiters}
\par Besides the most common delimiters like parentheses, brackets and braces there are many, there are some other commonly used paired delimiters, for example: \( \lvert x \rvert \) (or just \( |x| \)), \( \lVert x \rVert \) (or like this: \( \| x \| \)), \( \lfloor x \rfloor \), \( \lceil x \rceil\). Of course, \verb"\left" and \verb"\right" can be used with all there paired delimiters as well.\index{norm=double vertical bar=\lVert=\rVert,ceiling function=\lceil=\rceil,floor function=\lfloor=\rfloor,absolute value=vertical bar=\lvert=\rvert=\|}

\section{Exercise}
\begin{staticpart}
Typeset the following: \[\showSourceOnClick \sqrt[4]{\left\lvert\frac{a}{b}+\frac{c}{d}\right\rvert}=\frac{|ad+bc|^{1/4}}{|bd|^{1/4}}\]
\end{staticpart}
\[?\]

\section{Logical Symbols}
\par Cases are not rare when instead of words we use standard quantifiers of existence and universality and arrows to denote logical implication. There are commands for quantifiers \verb"\exists" and \verb"\forall". For arrows, there are commands with names just as telling (for those who know English) \verb"\implies" and \verb"\iff".\index{logical implication sign=if…then…=\implies,logical equivalence sign=if and only if=\iff}
\par Also useful are symbols denoting conjunction, disjunction, and negation: \verb"\land", \verb"\lor", \verb"\lnot"\index{disjunction=logical OR=\lor=\vee,conjunction=logical AND=\land=\wedge,logical negation=\lnot=\neg} (the letter \verb"l" before the names of commands for logical operations comes from the word \emph{logical}). An example of using all of the above: \[ \exists x \forall y (x\land y=x\lor y \iff \lnot y=\lnot x) \]
\par Two more remarks. First, the commands \verb"\land", \verb"\lor", \verb"\lnot" have synonym commands \verb"\wedge", \verb"\vee", \verb"\neg" --- choose the version to your taste. Second, commands \verb"\implies" and \verb"\iff", which display rather long arrows with margins on the sides, can be replaced with the commands \verb"\Rightarrow" and \verb"\Leftrightarrow", which produce shorter arrows without margins.\index{\Rightarrow,\Leftrightarrow}

\section{Arrows}
\par In the previous step, we became acquainted with the ``logical'' arrows \verb"\implies" and \verb"\iff" and their ``regular'' counterparts \verb"\Rightarrow" and \verb"\Leftrightarrow". It's not surprising that in \LaTeX there is also a command \verb"\Leftarrow". Notice that the last three mentioned commands all start with capital letters. Commands \verb"\leftarrow", \verb"\rightarrow" and \verb"\leftrightarrow" also exist and they output normal, ``thin'' arrows: \(\leftarrow,\rightarrow,\leftrightarrow\). The command \verb"\rightarrow" has another synonym command \verb"\to", introduced for brevity.\index{left arrow=\leftarrow,right arrow=\rightarrow=\to,bi-directional arrow=\leftrightarrow}

\par In \LaTeX arrows are very flexible, and you can do things like: \( f\xrightarrow{x\rightarrow 7}\infty \). As you can see, the \verb"\xrightarrow"\index{right arrow with label=\xrightarrow} command has one mandatory argument -- what should be displayed above the arrow. It will stretch horizontally automatically to fit everything necessary. By the way, \verb"\xrightarrow" also has an \emph{optional} argument -- what can be additionally displayed \emph{below} the arrow. Try it out yourself! And there is also a command \verb"\xleftarrow" -- here, I think, comments are unnecessary. Note that along the way we also got acquainted with the command for the infinity symbol!
\par In \LaTeX there are also universal commands \verb"\overset" and \verb"\underset", which place any symbols respectively above and below a given symbol. For example: \( a \overset{\Delta}{=} b \) and \( a \underset{c}{=} b \).\index{place above the symbol=\overset,place below the symbol=\underset} If you need to place something on both sides of a symbol, this is also doable: \( a \underset{!}{\overset{?}{=}} b \).
The following is about how arrows are typed using commands \verb"\rightarrow", \verb"\Rightarrow" and others (it's up to you to guess how to type similar arrows pointing to the left;)). There is a synonym for the command \verb"\rightarrow" -- it is the command \verb"\to", introduced for convenience: it's faster to type.

\par To put an arrow above an expression, you can use the command \verb"\overrightarrow"\index{\overrightarrow=arrow above expression}: \(\overrightarrow{abc}\). I think, you can guess what, for example, the command \verb"\underleftarrow" does ;)\index{\overleftarrow,\underleftarrow=arrow under expression,\underrightarrow=arrow under expression}

\section{Exercise}
\begin{staticpart}
Type the formulas: \[\showSourceOnClick  \forall x\forall y(\lnot (x\land y)=\lnot x\lor \lnot y) \] \[\showSourceOnClick  x^2 = y^2 \iff x=y \lor x+y=0 \]
\end{staticpart}
\[ ? \]
\[ ? \]

\section{How to Type a Mathematical Statement}
\par It's hardly possible to type a lot of mathematical text without encountering infinity symbols, quantifiers, and arrows. Here are examples that introduce you to all these things:
\[\lim_{x\rightarrow x_0}f(x)=f_0 \iff ( \forall \epsilon\in\mathbb{R}^+ \exists \delta\in\mathbb{R}^+: |x-x_0|<\epsilon \implies |f-f_0|<\delta ). \]
\par From this example, we learn that the symbol for an element belonging to a set is typed with the command \verb"\in". Also, in the analyzed example, we are introduced to the command \verb"\lim"\index{\lim}, which behaves interestingly: what we type after it as if in a subscript is placed under the word lim, as we are used to seeing in books. But this only happens in display mode. In inline mode, \LaTeX saves vertical space, and the command \verb"\lim" gives this result: \( \lim_{x\rightarrow x_0}f(x) \).
\par You might have noticed a couple of things: first, the letter \( \epsilon \) we are used to seeing in textbooks as this: \( \varepsilon \). Secondly, the lack of a space before the existence quantifier and not too large a space after the colon slightly impair the readability of the formula. Let's add spaces and the familiar epsilon, and replace the command \verb"\rightarrow" with its short synonym \verb"\to":
\[\lim_{x\to x_0}f(x)=f_0\quad \iff\quad ( \forall \varepsilon\in\mathbb{R}^+ \, \exists \delta\in\mathbb{R}^+:\: |x-x_0|<\varepsilon \implies |f-f_0|<\delta ). \]
We created spaces with the commands \verb"\quad" (inserts quadruple space), \verb"\," (inserts thin space) and \verb"\:" (inserts a space suitable after a colon).\index{\quad=quadruple space,thin space,\:}

\section{Mathematical Operators}
\par Mathematical operators that are abbreviations of Latin words are usually typed in upright font. Most often, in \LaTeX there is a command that does this, named the same as the operator itself: \verb"\lim", \verb"\sup", \verb"\min", and so on.\index{mathematical operators,\lim=limit,\sup=supremum,\inf=infimum,\min=minimum,\max=maximum} However, for some operators the corresponding command is not found. In this case, use the command \verb"\operatorname".\index{\operatorname} Here's an example: \[3=\operatorname{argmin}_{x\in\mathbb{N}}(x^2-8).\] The command \verb"\operatorname" has a sister - the command \verb"\operatorname*" (yes, in \LaTeX an asterisk can appear in the name of a command), which differs in how the "limits" of the operator are displayed in display-math mode. Put an asterisk yourself in the code in the example above, press \verb"Ctrl+Enter" and see what happens!
\par When you need to type just one letter in upright font, not being an operator by meaning, you can use the purely font-related command \verb"\mathrm" we discussed earlier.\index{\mathrm}
\par It would be strange not to mention operators like summation, here they are:
\[\sum_{k=0}^{100} k^2\]\index{\sum=sum}
\[\prod_{k=1}^{100} k\]\index{\prod=product}
\[\int_{-\pi}^\pi\sin x\,\mathrm{d}x.\]\index{\int=integral}
You can also experiment with the operators \verb"\bigcap", \verb"\bigcup", \verb"\bigsqcup", important for set-theoretical formulas.\index{intersection of sets=\bigcap,union of sets=\bigcup,disjoint union of sets=\bigsqcup}

\section{Exercise}
\begin{staticpart}
Type the text: \htmlblockquote{\par There exists a finite limit \[\showSourceOnClick \lim_{n\to\infty}\left(-\ln n + \sum_{k=1}^n \frac{1}{k}\right),\] named \emph{the Euler–Mascheroni constant} and enclosed within the range \(\showSourceOnClick [0.5,\,0.6]\).}
\end{staticpart}
\par There exists a finite limit \[?\] named the Euler–Mascheroni constant and enclosed within the range \(?\).

\section{Diacritical Marks}
\par ``X with a bar'', ``y with a hat'', ``z prime'', ``vector tau'' --- everything is possible: \(\bar{x},\, \hat{y},\, z',\, \vec{\tau}\).\index{diacritical marks,prime on symbol,hat on symbol=\hat,bar on symbol=overline=\bar=\overline,tilde on symbol=wave on symbol=\tilde=\widetilde} A full list of standard commands for adding ``headgear'' to symbols can be seen \href{https://en.wikibooks.org/wiki/LaTeX/Special_Characters#Math_mode}{at this link}.
\par What you certainly need to know is the following. When you need to ``hang a hat'' on a long expression, you should use ``long'' versions of commands:\index{\hat=\widehat,\vec=\overrightarrow,\tilde=\widetilde}
\[\tilde{abcd} \text{ vs. } \widetilde{abcd}\]
\[\hat{abcd} \text{ vs. } \widehat{abcd}\]
\[\bar{abcd} \text{ vs. } \overline{abcd}\]
\[\vec{abcd} \text{ vs. } \overrightarrow{abcd}\]
\par Note: to include accompanying text ``vs.'' that does not carry direct mathematical significance in the body of the formula, the command \verb"\text" is used.\index{placing text inside formulas=\text}

\section{Underlining and Overlining}
\par Let's learn by example: \[\overline{abc}\quad \underline{abc}\quad \underbrace{abc}_{\text{explanation}} \quad \overbrace{abc}^{\text{explanation}}\]\index{line under symbol=underlining=\underline,line over symbol=overlining=\overline,curly brace over expression=\overbrace,curly brace under expression=\underbrace}
Notice that when textual explanations are needed within formulas, they are enclosed in the command \verb"\text"\index{placing text inside formulas=\text}, which outputs what is passed to it in text, not mathematical mode. Inside the \verb"\text" command, you can temporarily switch to mathematical mode:
\[ \underbrace{a^3=abc}_{\text{because \(a^2=bc\)}} \]

\section{Exercise}
\begin{staticpart}
Type the text: \htmlblockquote{\par A number, written in binary as: \[\showSourceOnClick  \underbrace{11\ldots 1}_{\text{$k$ ones}},\] equals \(2^k-1\). By definition, we assume \(\showSourceOnClick 2^k\overset{\mathrm{def}}{=} \overbrace{2\cdot 2\cdot\ldots \cdot 2}^{\text{$k$ times}}\).}
\end{staticpart}
\par A number, written in binary as: \[11\ldots 1,\] equals \(2^k-1\).
By definition, we assume \(2^k=2 \cdot 2 \cdot\ldots\cdot 2\).

\section{Inequalities and \(O\)-Notation}
\par Equality and strict inequalities in \LaTeX are denoted by the symbols that are directly on the keyboard. For non-strict inequalities, there are commands \verb"\le" and \verb"\ge". But more customary for, for example, Russian typography, are the symbols produced by commands \verb"\leqslant" and \verb"\geqslant". Compare: \(\le,\ge,\leqslant,\geqslant\).

\section{Calculations Modulo}
\par In the \LaTeX system, there are three commands that output the word \emph{mod} in different variations. It is easiest to consider an example:\index{mod=\mod=\pmod=\bmod=modulo=remainder of division}
\begin{itemize}
\item Consider the number \(c\), equal to \(a\bmod b\).
\item The following equality holds: \[a=c\pmod{b}\]
\item The following equality holds: \[a=c\mod b\]
\end{itemize}
