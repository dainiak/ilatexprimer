\section{Indices and powers (subscripts and superscripts)}
\par Perhaps, the first thing we should learn to typeset are the indices.\index{lower index=subscript=_}
Subscript index can be typeset as follows: \(a_i\).
And here is the superscript: \(e^x\).\index{upper index=power=superscript=^}
Note that when there is more than one symbol in the index, it cannot be just left as is, or we get: $a_index$. To make this typeset properly, we need to group it with curly braces: $a_{index}$.
One can combine subscript and superscript: $a_i^2$ or $a^2_i$ --- if you look at the code you will see that the typeset result does not depend on the order in the source code. Of course, it’s up to you to choose what comes first, yet I recommend to start with the index that is an essential part of the object notation. For instance, if the upper index designates the power, it is recommended to write \verb"a_i^2".

\section{Greek letters}
\par Besides latin letters $a$, $b$, etc, there is a lot of use for the Greek letters in mathematics. They are all typeset using the commands which coincide with the letter names. Also, the command to typeset capital Greek letters start with capital letters themselves:
\[\alpha, \beta, \gamma, \delta, \xi, \Delta, \Omega\]\index{Greek letters=Greek alphabet}
and so on. The list of all the commands for Greek letter typesetting can be found \href{https://www.latex-tutorial.com/symbols/greek-alphabet/}{here}.

\section{Math mode fonts}
\par OK, now that we’ve learned how to typeset Greek letters, what do we do if we want to typeset a latin letter, but with some special font? One of the common cases is the typesetting of the number sets \(\mathbb{N}, \mathbb{Z}, \mathbb{Q}, \mathbb{R}\)\index{blackboard bold mathematical font=\mathbb}. Or, for instance, roman “non-italic” font for probability measure operator \(\mathrm{P}\) or for operators \(\mathrm{diam}, \mathrm{argmin}\)\index{math roman font=math straight font=math non-italic font=\mathrm}, etc. Sometimes we need the following “calligraphic”, “hand-written” letters: \(\mathcal{A}\)\index{\mathcal=calligraphic mathematical font=hand-written mathematical font} or, perhaps, an even more resembling the hand-writing \(\mathscr{A}\)\index{\mathscr=calligraphic mathematical font=hand-written mathematical font}. Somewhat more rare, but still used, is the gothic font: \( \mathfrak{B} \)\index{\mathfrak=gothic font}. Well, if you have looked into the \LaTeX-code of the current section, you now know how to make \LaTeX typeset that: with commands \verb"\mathbb", \verb"\mathrm", \verb"\mathcal", \verb"\mathscr" and others. A more comprehensive list of font commands can be found \href{http://tex.stackexchange.com/a/58124}{here}. Oh, boy, that’s a lot of commands just for setting the font! But don’t worry, it is all memorized when actually being used. For now, I recommend having the printed list of commands that you believe to be useful to you in front of you when you typeset your first texts. In a month of actively using \LaTeX you are likely to no longer need this list;)

\section{Exercise}
\begin{staticpart}
Typeset the following:\htmlblockquote{\par If one takes two rational numbers \(\showSourceOnClick \alpha\) and \(\showSourceOnClick \beta\) from the set of rationals \(\showSourceOnClick \mathbb{Q} \), then the number \(\showSourceOnClick (\alpha+\beta)^2\) is also going to be rational. By the way, one can find a number in \(\showSourceOnClick \mathbb{R}\), say \(\showSourceOnClick \gamma\), such that \(\showSourceOnClick \gamma^2\) is rational, but the \(\showSourceOnClick \gamma\) is not. Think of how to find two \emph{irrational} numbers \(\showSourceOnClick \alpha_1\) and \(\showSourceOnClick \alpha_2\), such that the number \(\showSourceOnClick \alpha_1^{\alpha_2}\) is \emph{rational}.}
\end{staticpart}
\par If one takes two rational numbers ??? and ??? from the set of rationals ???, then the number ??? is also going to be rational. By the way, one can find a number in ???, say ???, such that ??? is rational, but the ??? is not. Think of how to find two irrational numbers ??? and ???, such that the number ??? is rational.

\section{Roots}
\par In this step I strongly suggest you to look at the source code alongside the text. The square root symbol can be typeset as follows: \( \sqrt{2} \).\index{square root=\sqrt} If you need cubic or any other root, employ the optional parameter of \verb"\sqrt": \( \sqrt[3]{2} \). By the way, in the first case you could omit the braces and still get the same result: \( \sqrt 2 \), but when you have a root of some multi-digit number, there might be a problem with the latter way: \( \sqrt 25 \) does not go. This is a fine moment to introduce the way in which \LaTeX commands work. The commands are needed to typeset almost anything beyond single numbers and letters: roots, fractions, and many others. A command may have several required arguments and several (usually zero or one) optional arguments. For instance, the command \verb"\sqrt" has one required argument: what to put under the root symbol. The optional argument for this command is the degree of the root. Optional arguments are passed in the square brackets.\index{optional arguments of commands} Required arguments either go after the name of the command (after an additional space in this case) or are provided (and grouped at the same time) in braces. Most of the time, explicit grouping is needed, so I suggest \emph{always} keeping the braces present around the required arguments.
\par It should be noted separately that in mathematical articles the roots of degree more than 2 are typically typeset as powers. For instance, in professional mathematical texts it is much more common to see \( 2^{1/3} \) than \( \sqrt[3]{2} \).

\section{Fractions and binomial coefficients}
\par In \LaTeX fractions are typeset using \verb"\frac" command with two required arguments --- the numerator and the denominator of the fraction: \( \frac{a}{b} \). The command \verb"\binom" which is used to typeset the symbol of binomial coefficient behaves analogously: \( \binom{n}{k} = \binom{n}{n-k} \).\index{fraction=\frac=\over,binomial coefficient=\binom=\choose}
\par For completeness I need to note that there are also \TeX commands \verb"\over" and \verb"\choose" that are employed in the infix style to typeset fractions and binomial coefficients, but the usage of these commands is not recommended in \LaTeX.

\section{Delimiters: parentheses, brackets, braces, etc.}
\par Delimiters of different types play quite an important role in \LaTeX as you have had chance to see in the previous steps: curly braces group content, and square brackets are used to provide optional arguments to the commands. In fact, square brackets are usually displayed normally: \( [a,b] \), but to display curly braces one has to prepend them with a backslash symbol: \( \mathbb{N} = \{1,2,3,\ldots\} \). Without that we would get: \( \mathbb{N} = {1,2,3,\ldots} \). As for the ordinary parentheses, they are always displayed normally and do not play any special role in \LaTeX coding.
\par One important thing to learn is stretching delimiters. For instance, suppose we need to raise a fraction into some power: \( (\frac{x^4}{a^b})^7 \). As you can see, the fraction ``outgrows'' the parentheses that surround it, and in display math mode the dissonance is even more prominent: \[ (\frac{x^4}{a^b})^7 .\]
To stretch the parentheses for them to suite the fraction height, we need to put a special command \verb"\left"\index{stretching delimiters=\left=\right} before the opening parenthesis and \verb"\right" before the closing one. Just look at how much better it is now: \[ \left(\frac{x^4}{a^b}\right)^7 .\]
Of course, the pair \verb"\left…\right" can be used with brackets and braces in the same manner.
\par As gorgeous as they are, \verb"\left" and \verb"\right" should not be used everywhere, see the details \href{http://tex.stackexchange.com/a/58641}{here}. The rule of thumb is as follows: plug in \verb"\left" and \verb"\right" when you see prominent difference between the delimiters and the height of the expression which is surrounded by them.
\par A perfectionist author may not be happy with just \verb"\left" and \verb"\right". For instance, in \[\left( 1 + \left( \left( a+b \right)/\left(a-b\right) \right) \right)^5\] it would be great if the outermost parentheses were larger than the inner ones, but as the height of all the expressions is the same, the \verb"\left" and \verb"\right" commands are simply useless in this case, as they have no reason to alter the default delimiter height. So we have a reason to employ the new commands \verb"\bigl", \verb"\Bigl", \verb"\biggl" and \verb"\Biggl"\index{\bigl,\Bigl,\biggl,\Biggl,\bigr,\Bigr,\biggr,\Biggr} and their ``right-handed counterparts'' with letter r at the end: \[\Bigl( 1 + \bigl( (a+b)/(a-b) \bigr) \Bigr)^5.\]
These ``big-commands'' are also useful when \verb"\left" and \verb"\right" are overreacting and the parentheses become the dominant symbols in the whole expression: of the two formulas
\[
\left( \sum_{i=1}^{n-1} i \right)
\quad\text{vs.}\quad
\biggl(\sum_{i=1}^{n-1} i\biggr)
\]
the second one looks better.

\section{More on delimiters}
\par Besides the most common delimiters like parentheses, brackets and braces there are many, there are some other commonly used paired delimiters, for example: \( \lvert x \rvert \) (or just \( |x| \)), \( \lVert x \rVert \) (or like this: \( \| x \| \)), \( \lfloor x \rfloor \), \( \lceil x \rceil\). Of course, \verb"\left" and \verb"\right" can be used with all there paired delimiters as well.\index{norm=double vertical bar=\lVert=\rVert,ceiling function=\lceil=\rceil,floor function=\lfloor=\rfloor,absolute value=vertical bar=\lvert=\rvert=\|}

\section{Exercise}
\begin{staticpart}
Typeset the following: \[\showSourceOnClick \sqrt[4]{\left\lvert\frac{a}{b}+\frac{c}{d}\right\rvert}=\frac{|ad+bc|^{1/4}}{|bd|^{1/4}}\]
\end{staticpart}
\[?\]

\section{Logical symbols}
\par It is also not uncommon to substitute standard logical symbols instead of words to denote implication and logical equivalence. Нередки случаи, когда вместо слов мы используем стандартные кванторы существования и всеобщности и стрелки для обозначения логического следования. Для кванторов есть команды \verb"\exists" и \verb"\forall". Для стрелок есть команды с настолько же говорящими (для знающих английский) именами \verb"\implies" и \verb"\iff".\index{знак логического следования=если…то…=\implies,критерий=знак логического тождества=тогда и только тогда=\iff}
\par Также могут пригодиться и символы, обозначающие конъюнкцию, дизъюнкцию и отрицание: \verb"\land", \verb"\lor", \verb"\lnot"\index{дизъюнкция=логическое ИЛИ=\lor=\vee,конъюнкция=логическое И=\land=\wedge,логическое отрицание=\lnot=\neg} (буква \verb"l" перед названиями команд для логических операций возникла от слова \emph{logical}). Пример использования всего перечисленного: \[ \exists x \forall y (x\land y=x\lor y \iff \lnot y=\lnot x) \]
\par Ещё два замечания. Во-первых, у команд \verb"\land", \verb"\lor", \verb"\lnot" есть команды-синонимы \verb"\wedge", \verb"\vee", \verb"\neg" --- выбирайте вариант на свой вкус. Во-вторых, командам \verb"\implies" и \verb"\iff", отображающим довольно длинные стрелки с отступами по бокам, альтернативой могут служить команды \verb"\Rightarrow" и \verb"\Leftrightarrow", выводящие стрелки покороче и без отступов.\index{\Rightarrow,\Leftrightarrow}

\section{Стрелки}
\par На предыдущем шаге мы познакомились с <<логическими>> стрелками \verb"\implies" и \verb"\iff" и их <<обычными>> аналогами \verb"\Rightarrow" и \verb"\Leftrightarrow". Неудивительно, что в \LaTeX есть ещё команда \verb"Leftarrow". Заметьте, что три последние упомянутые команды все начинаются с заглавных букв. Команды \verb"\leftarrow", \verb"\rightarrow" и \verb"\leftrightarrow" тоже есть и выводят они обычные, <<тонкие>> стрелки: \(\leftarrow,\rightarrow,\leftrightarrow\). У команды \verb"\rightarrow" есть ещё команда-синоним \verb"\to", введённая для краткости.\index{стрелка влево=\leftarrow,стрелка вправо=\rightarrow=\to,стрелка в обе стороны=\leftrightarrow}

\par В \LaTeX стрелки очень гибки, и можно делать такие вещи: \( f\xrightarrow{x\rightarrow 7}\infty \). Как видно, у команды \verb"\xrightarrow"\index{стрелка вправо с подписью=\xrightarrow} один обязательный аргумент --- это то, что должно быть отображено над стрелкой. Растянется она по горизонтали автоматически, чтобы вместить всё необходимое. Кстати, у \verb"\xrightarrow" есть ещё и \emph{необязательный} аргумент --- то, что может быть дополнительно изображено \emph{под} стрелкой. Попробуйте самостоятельно его в действии! А ещё есть команда \verb"\xleftarrow" --- тут, думаю, комментарии излишни. Заметьте, что по ходу дела мы и с командой для значка бесконечности познакомились!
\par В \LaTeX имеются ещё универсальные команды \verb"\overset" и \verb"\underset", которые размещают любые символы соответвенно над и под заданным значком. Например: \( a \overset{\Delta}{=} b \) и \( a \underset{c}{=} b \).\index{разместить сверху от значка=\overset,разместить снизу от значка=\underset} Если нужно что-разместить с обеих сторон значка, это тоже выполнимо: \( a \underset{!}{\overset{?}{=}} b \).
Следующее --- что стрелки набираются командами \verb"\rightarrow", \verb"\Rightarrow" и другими (угадайте, как набрать стрелки, смотрящие влево;)). Есть синоним для команды \verb"\rightarrow" --- это команда \verb"\to", введённая для удобства: набирать её быстрее.

\par Чтобы поставить стрелку над выражением, можно воспользоваться командой \verb"\overrightarrow"\index{\overrightarrow=стрелка над выражением}: \(\overrightarrow{abc}\). Думаю, Вы сами догадаетесь, что, скажем, делает команда \verb"\underleftarrow";)\index{\overleftarrow,\underleftarrow=стрелка под выражением,\underrightarrow=стрелка под выражением}

\section{Упражнение}
\begin{staticpart}
Наберите формулы: \[\showSourceOnClick  \forall x\forall y(\lnot (x\land y)=\lnot x\lor \lnot y) \] \[\showSourceOnClick  x^2 = y^2 \iff x=y \lor x+y=0 \]
\end{staticpart}
\[ ? \]
\[ ? \]

\section{Как набрать формулировку из матана}
\par Вряд ли можно набрать много математического текста и не столкнуться со значками бесконечности, кванторами и стрелками. Вот примеры, которые знакомят Вас со всеми этими вещами:
\[\lim_{x\rightarrow x_0}f(x)=f_0 \iff ( \forall \epsilon\in\mathbb{R}^+ \exists \delta\in\mathbb{R}^+: |x-x_0|<\epsilon \implies |f-f_0|<\delta ). \]
\par Из этого примера мы узнаём, что значок принадлежности элемента множеству набирается командой \verb"\in". Ещё в разобранном примере мы знакомимся с командой \verb"lim"\index{\lim}, которая ведёт себя интересным образом: то, что после неё мы набираем как бы в нижнем индексе, заносится вообще под слово lim, как мы и привыкли видеть в книжках. Но это происходит только в display-режиме. В inline-режиме \LaTeX экономит вертикальное место и команда \verb"\lim" даст такой результат: \( \lim_{x\rightarrow x_0}f(x) \).
\par Вы могли обратить внимание ещё на пару вещей: во-первых, букву \( \epsilon \) мы привыкли видеть в учебниках другой, а именно, такой: \( \varepsilon \). Во-вторых, отсутствие отступа перед квантором существования и не слишком большой отступ после двоеточия несколько ухудшают читабельность формулы. Давайте добавим отступы и привычный эпсилон, а команду \verb"\rightarrow" заменим на короткий синоним \verb"\to":
\[\lim_{x\to x_0}f(x)=f_0\quad \iff\quad ( \forall \varepsilon\in\mathbb{R}^+ \, \exists \delta\in\mathbb{R}^+:\: |x-x_0|<\varepsilon \implies |f-f_0|<\delta ). \]
Отступы мы создали командами \verb"\quad" (вставляет учетверённый пробел), \verb"\," (вставляет узкий пробел) и \verb"\:" (вставляет пробел, подходящий после двоеточия).\index{\quad=четверной пробел,узкий пробел,\:}

\section{Математические операторы}
\par Математические операторы, являющиеся сокращениями от латинских слов, принято набирать прямым шрифтом. Чаще всего, в \LaTeX находится команда, которая это делает, называющаяся так же, как и сам оператор: \verb"\lim", \verb"\sup", \verb"\min", и так далее.\index{математические операторы,\lim=предел,\sup=супремум,\inf=инфимум,\min=минимум,\max=максимум} Однако для некоторых операторов соответствующая команда не находится. В этом случае используйте команду \verb"\operatorname".\index{\operatorname} Вот пример: \[3=\operatorname{argmin}_{x\in\mathbb{N}}(x^2-8).\] У команды \verb"\operatorname" есть сестра --- команда \verb"\operatorname*" (да-да, в \LaTeX звёздочка может быть полноценным символом в имени команды), которая отличается тем, как будут отображаться «пределы» у оператора в режиме display-math. Поставьте сами звёздочку в коде в примере выше, нажмите \verb"Ctrl+Enter" и увидите, что получится!
\par Когда нужно набрать прямым шрифтом лишь одну букву, не являющуюся по смыслу оператором, можно использовать рассмотренную нами ранее чисто шрифтовую команду \verb"\mathrm".\index{\mathrm}
\par Было бы странно не упомянуть об операторах типа суммирования, вот они:
\[\sum_{k=0}^{100} k^2\]\index{\sum=сумма}
\[\prod_{k=1}^{100} k\]\index{\prod=произведение}
\[\int_{-\pi}^\pi\sin x\,\mathrm{d}x.\]\index{\int=интеграл}
Можете также поэкспериментировать с операторами \verb"\bigcap", \verb"\bigcup", \verb"\bigsqcup", важными для теоретико-множественных формул.\index{пересечение множеств=\bigcap,объединение множеств=\bigcup,дизъюнктное объединение множеств=\bigsqcup}

\section{Упражнение}
\begin{staticpart}
Наберите текст: \htmlblockquote{\par Существует конечный предел \[\showSourceOnClick \lim_{n\to\infty}\left(-\ln n + \sum_{k=1}^n \frac{1}{k}\right),\] называемый \emph{постоянной Эйлера--Маскерони} и заключённый в диапазоне \(\showSourceOnClick [0.5,\,0.6]\).}
\end{staticpart}
\par Существует конечный предел \[?\] называемый постоянной Эйлера--Маскерони и заключённый в диапазоне \(?\).

\section{Диакритические знаки}
\par «Икс с чертой», «игрек с крышкой», «зет штрих», «вектор тау» --- всё возможно: \(\bar{x},\, \hat{y},\, z',\, \vec{\tau}\).\index{диакритические знаки,штрих у символа,крышка над символом=\hat,черта над символом=надчёркивание=\bar=\overline,тильда над символом=волна над символом=\tilde=\widetilde} Полный список стандартных команд для добавления символам «головных уборов» можно посмотреть \href{https://en.wikibooks.org/wiki/LaTeX/Special_Characters#Math_mode}{по ссылке}.
\par Знать ещё обязательно нужно, пожалуй, только следующее. Когда нужно «навесить крышу» на длинное выражение, нужно использовать <<длинные>> варианты команд:\index{\hat=\widehat,\vec=\overrightarrow,\tilde=\widetilde}
\[\tilde{abcd} \text{ vs. } \widetilde{abcd}\]
\[\hat{abcd} \text{ vs. } \widehat{abcd}\]
\[\bar{abcd} \text{ vs. } \overline{abcd}\]
\[\vec{abcd} \text{ vs. } \overrightarrow{abcd}\]
\par Обратите внимание: для того, чтобы включить в тело формулы сопроводительный текст ``vs.'', не несущий непосредственной математической нагрузки, используется команда \verb"\text".\index{размещение текста внутри формулы=\text}

\section{Подчёркивание и надчёркивание}
\par Учимся на примере: \[\overline{abc}\quad \underline{abc}\quad \underbrace{abc}_{\text{пояснение}} \quad \overbrace{abc}^{\text{пояснение}}\]\index{черта под символом=подчёркивание=\underline,черта над символом=надчёркивание=\overline,фигурная скобка над выражением=\overbrace,фигурная скобка под выражением=\underbrace}
Заметьте, что когда внутри формулы нужно дать текстовые пояснения, их заключают в команду \verb"\text"\index{размещение текста внутри формулы=\text}, которая выводит то, что ей передают, в текстовом, а не математическом режиме. Внутри команды \verb"\text" можно временно перейти в математический режим:
\[ \underbrace{a^3=abc}_{\text{потому что \(a^2=bc\)}} \]

\section{Упражнение}
\begin{staticpart}
Наберите текст: \htmlblockquote{\par Число, записываемое в двоичной системе так: \[\showSourceOnClick \underbrace{11\ldots 1}_{\text{$k$ единиц}},\] равняется \(2^k-1\). По определению мы полагаем \(\showSourceOnClick 2^k\overset{\mathrm{def}}{=} \overbrace{2\cdot 2\cdot\ldots \cdot 2}^{\text{$k$ раз}}\).}
\end{staticpart}
\par Число, записываемое в двоичной системе так: \[11\ldots 1,\] равняется \(2^k-1\).
По определению мы полагаем \(2^k=2 \cdot 2 \cdot\ldots\cdot 2\).

\section{Неравенства и \(O\)-символика}
\par Равенство и строгие неравенства в \LaTeX обозначаются теми значками, которые непосредственно есть на клавиатуре. Для нестрогих неравенств есть команды \verb"\le" и \verb"\ge". Но более привычны для нас, россиян, значки, выводимые командами \verb"\leqslant" и \verb"\geqslant". Сравните: \(\le,\ge,\leqslant,\geqslant\).

\section{Вычисления по модулю}
\par В системе \LaTeX есть целых три команды, которые выводят слово \emph{mod} в разных вариациях. Проще всего рассмотреть пример:\index{mod=\mod=\pmod=\bmod=по модулю=остаток от деления}
\begin{itemize}
\item Рассмотрим число \(c\), равное \(a\bmod b\).
\item Имеет место равенство: \[a=c\pmod{b}\]
\item Имеет место равенство: \[a=c\mod b\]
\end{itemize}