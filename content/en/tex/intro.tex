\section{Important study recommendations (click here to view) ←}
\par This lesson is interactive: the text of the lesson is prepared with \LaTeX, and you should be able now to see the source code to the right of the displayed content. When you hover the mouse over any formula except the ones in the exercises, you will see the source code for that formula. Pay attention to the way each formula is coded in LaTeX — this is a crucial element of learning. At any moment you can modify the source code on the right and see the result of this modification on the left either immediately or upon pressing \verb"Ctrl+Enter" (depends on your settings). So feel free to experiment right here, with the lesson’s source. Note still, that this lesson is far from being a fully-featured \LaTeX distribution, so to experience the full power of \LaTeX you will need to either install a \LaTeX compiler locally, or to use a cloud service, such as \href{https://www.overleaf.com/}{Overleaf} or \href{https://papeeria.com/landing}{Papeeria}. I will keep you informed when you need to use a full compiler.
\par The lesson consists of small “steps” explaining some aspects of typesetting formulas in \LaTeX. These steps are interleaved with exercises. The latter ones are a hugely important element of the lesson, as just like any programming language, \LaTeX is best learnt by actually doing things with it. If (and only if) you face significant difficulties while doing an exercise, you can always view the source code for the formula by clicking it with a mouse.
\par Last, but not least, if you have already worked through this lesson and want just to quickly look thing up, you can use the smart search feature. When you enter some \LaTeX command, all steps containing this command will be highlighter and you will be able to see how the command is used in context.