\section{Important study recommendations (click here to view) ←}
\par This lesson is interactive: the lesson is written itself in \LaTeX, and you should be able now to see the sourse code to the right of the displayed content. When you hover the mouse over any formula except the one in the exercise, you will see the source code for that formula. Keep attention to the way each formula is coded in LaTeX — this is a crucial element of learning. Moreover, at any moment you can modify the source code on the right and see the result of this modification on the left upon pressing \verb"Ctrl+Enter". So feel free to experiment right here, with the lesson’s sourse. Note still, that this is not a complete \LaTeX environment, so to feel the full power of \LaTeX you will need to either install a compiler locally, or use a cloud service, such as \href{https://papeeria.com/landing}{Papeeria}, \href{https://www.sharelatex.com/}{ShareLaTeX} or \href{https://www.overleaf.com/index_b}{Overleaf}. I will keep you informed when you need to use a full compiler.
\par The lesson consists of small “steps” explaining some aspects of writing in \LaTeX. These steps are interleaved with exercises. The latter ones are a hugely important element of the lesson, as just with any programming language, \LaTeX is best learnt by actually doing things. If (and only if) you face significant difficulties when doing an exercise, you can view the source code for the formula by clicking it with a mouse.
\par If you have already worked through this lesson and want just to quickly lool thing up, you can use the keyword smart search. When you enter some \LaTeX command, all steps containing this command will be opened and you will be able to see how it is used in context.